% !TEX encoding = UTF-8
\addcontentsline{toc}{chapter}{附录A~~本论文相关源代码}%添加到目录中
\chapter*{附录A~~~~本论文相关源代码}
\appedixfigtabnum{A}%重新计算附图和表的标题号和计数号,参数是附录A,B或者C...
这里是附录页,附上你的程序或必要的相关知识

\bf\heiti\color{red}若要生成目录和参考文献的编译方式:

\color{black}XeLaTeX --> BibTeX --> XeLaTeX --> XeLaTeX

\fangsong
对于一些不宜放入正文中、但作为毕业论文(设计)又是不可缺少的部分,或有重要参考价值的内容,可编入毕业论文(设计)的附录中。例如,过长的公式推导、重复性的数据、图表、程序全文及其说明等。论文的附录依序用大写正体A,B,C……编序号,如:附录A。附录中的图、表、式等另行编序号,与正文分开,也一律用阿拉伯数字编码,但在数码前冠以附录序码,如:图A1;表B2;式(B3)等,

这个示例为插入图片:
\begin{figure}[H]
	\centering
	\includegraphics[width=0.618\textwidth]{figure.jpg}%图片名称,放在/figures目录下
	\caption{图片插入\label{fig:appA}}
\end{figure}

\begin{table}[H]
	\begin{center}
		\caption{希腊字母表\label{tab:appA}}
		\begin{tabular}{|c|c|c|c|c|}
			\hline
			Alpha & Beta & Gamma & Delta & Theta\\
			\hline
			$\alpha$ & $\beta$ & $\gamma$ & $\delta$ & $\theta$\\
			\hline
			$A$ & $B$ & $\Gamma$ & $\Delta$ & $\Theta$\\
			\hline
		\end{tabular}
	\end{center}
\end{table}

\begin{equation}
	\hat{H}=\frac{\epsilon}{2}\hat{\sigma}_{z}-\frac{\Delta}{2}\hat{\sigma}_{x}+\sum_{k}\omega_{k}\hat{b}_{k}^{\dagger}\hat{b}_{k}+\sum_{k}\frac{g_{k}}{2}\hat{\sigma}_{z}(\hat{b}_{k}+\hat{b}_{k}^{\dagger})\label{eq:appA}
\end{equation}

\addcontentsline{toc}{chapter}{附录B~~其他需要放到附录里的东西}%添加到目录中
\chapter*{附录B~~~~其他需要放到附录里的东西}
\appedixfigtabnum{B}%重新计算附图和表的标题号和计数号
这个示例为插入图片:
\begin{figure}[H]
	\centering
	\includegraphics[width=0.618\textwidth]{figure.jpg}%图片名称,放在/figures目录下
	\caption{图片插入\label{fig:appB}}
\end{figure}

\begin{table}[H]
	\begin{center}
		\caption{希腊字母表\label{tab:appB}}
		\begin{tabular}{|c|c|c|c|c|}
			\hline
			Alpha & Beta & Gamma & Delta & Theta\\
			\hline
			$\alpha$ & $\beta$ & $\gamma$ & $\delta$ & $\theta$\\
			\hline
			$A$ & $B$ & $\Gamma$ & $\Delta$ & $\Theta$\\
			\hline
		\end{tabular}
	\end{center}
\end{table}

\begin{equation}
	\hat{H}=\frac{\epsilon}{2}\hat{\sigma}_{z}-\frac{\Delta}{2}\hat{\sigma}_{x}+\sum_{k}\omega_{k}\hat{b}_{k}^{\dagger}\hat{b}_{k}+\sum_{k}\frac{g_{k}}{2}\hat{\sigma}_{z}(\hat{b}_{k}+\hat{b}_{k}^{\dagger})\label{eq:appB}
\end{equation}